\documentclass{article}
\usepackage[utf8]{inputenc}

\date{30 avril 2017}

\usepackage{natbib}
\usepackage{graphicx}
\usepackage[francais]{babel}
\setlength{\parskip}{\baselineskip}

%%\begin{document}

%%%%%%%%%%%%%%%%%%%%%%%%%%%%%%%%%%%%%%%%%
%\title{Title page with logo}
%----------------------------------------------------------------------------------------
%	PACKAGES AND OTHER DOCUMENT CONFIGURATIONS
%----------------------------------------------------------------------------------------

\documentclass[12pt]{article}
\usepackage[english]{babel}
\usepackage[utf8x]{inputenc}
\usepackage{amsmath}
\usepackage{graphicx}
\usepackage[colorinlistoftodos]{todonotes}

\begin{document}

\begin{titlepage}

\newcommand{\HRule}{\rule{\linewidth}{0.5mm}} % Defines a new command for the horizontal lines, change thickness here

\center % Center everything on the page
 
%----------------------------------------------------------------------------------------
%	HEADING SECTIONS
%----------------------------------------------------------------------------------------

\textsc{\LARGE École Nationale des Ponts et Chaussées}\\[1.5cm] % Name of your university/college
\textsc{\Large Project Workshop}\\[0.5cm] % Major heading such as course name

%----------------------------------------------------------------------------------------
%	TITLE SECTION
%----------------------------------------------------------------------------------------

\HRule \\[0.4cm]
{ \huge \bfseries An analysis of 'Black Mirror'}\\[0.4cm] % Title of your document
\HRule \\[1.5cm]
 
%----------------------------------------------------------------------------------------
%	AUTHOR SECTION
%----------------------------------------------------------------------------------------

\begin{minipage}{0.4\textwidth}
\begin{flushleft} \large
\emph{Authors:}\\
Mathieu \textsc{Orhan}\\ Louis \textsc{Montaut} % Your name
\end{flushleft}
\end{minipage}
~
\begin{minipage}{0.4\textwidth}
\begin{flushright} \large
\emph{Teacher:} \\
Thomas \textsc{Harcharick} % Supervisor's Name
\end{flushright}
\end{minipage}\\[2cm]

% If you don't want a supervisor, uncomment the two lines below and remove the section above
%\Large \emph{Author:}\\
%John \textsc{Smith}\\[3cm] % Your name

%----------------------------------------------------------------------------------------
%	DATE SECTION
%----------------------------------------------------------------------------------------

{\large \today}\\[2cm] % Date, change the \today to a set date if you want to be precise

%----------------------------------------------------------------------------------------
%	LOGO SECTION
%----------------------------------------------------------------------------------------

\includegraphics{logo.jpg}\\[1cm] % Include a department/university logo - this will require the graphicx package
 
%----------------------------------------------------------------------------------------

\vfill % Fill the rest of the page with whitespace

\end{titlepage}

\section*{Abstract}\\
'Black Mirror" is a dystopian and technology-driven anthology. The concept of this British TV show was designed by the  journalist, actor and screenwriter Charlie Brooker in 2011, and three short seasons had been broadcasted yet. Its title, 'Black Mirror', obviously refers to the reflective black displays surrounding us. Each episode describes the unexpected side-affects of a current or a potential technology. Our addiction towards new technologies and their effects on us is depicted through dark and satirical stories.
\begin{center}
\includegraphics[width=12cm]{Black-Mirror-logo.jpg}
\end{center}\\
We chose to talk about a dystopian TV-show because we hold dear to our hearts the dystopian genre. To sum-up, a dystopian world, just like an utopian world, is a world similar to ours, usually in the future, where certain characteristics have been amplified to show their effects. In a dystopia, these effects usually turn out to be bad.
Therefore, a dystopia is a way of letting us question about certain aspects of our world to anticipate what they might become. Many fruitfull debates emerge from a dystopia analysis.

In the following sections, we will describe and analyse the six first episodes of this show.


\newpage

\section{First season}

\subsection{The National Anthem}
Among Black Mirror's episodes, The National Anthem is maybe one of the most provocative and weird, despite the fact that it takes place in our actual world. 
It is the least dystopic episode. Yet, we can extract many thoughts from it.


\begin{center}
\includegraphics[width=12cm]{s1e1.jpg}
\caption{The Prime Minister}
\end{center}\\

\textbf{Plot} \\
The prime minister, Michael Callow, is notified early in the morning by its advisers that the popular "princess Susannah", the duchess of Beaumont, has been kidnapped. The ransom concerns him directly, and is both surprising and disgusting. To free the princess, the unknown abductor demands that he have sex with a pig, without any special effects, and broadcast it live on all British TV channels before 16pm. \\
The prime minister is disgusted and tries to silence the media, but the information is indeed on Youtube. The news spreads over the Internet, because in the aftermath of removing one video, copies of it are uploaded again. Finally, with the social media pressure, the British newspapers break the silence, but the public opinion is behind the first minister. \\
Without him knowing, the authorities hire a pornographic actor to green-screen the prime minister head onto him. Meanwhile, they locate the abductor and engage military actions, which is a failure. When the green-screen trick is discovered, the abductor send a cut finger to the media, making the public opinion shift. With the increasing pressure, he ultimately decides to do it.\\
Everyone watches television, but in the empty city, the princess collapses in the middle of a bridge thirty minutes before the deadline.
Most people are horrified but they still watch it. In fact, the princess is unharmed, the finger belongs to the abductor. It was meant to be an artistic performance and its author committed suicide. One year later, the prime minister is very popular but in private, his wife is cold and aloof towards him.

\begin{center}
\includegraphics[width=12cm]{s1e1b.jpg}
\end{center}

\textbf{Analysis}\\
This episode focuses on many unhealthy aspects of social network and social medias.\\
First of all, this episode denounces that the political life of politician is deeply bounded with social networks. It comes with a good side of the coin and a negative side. The good side which is also the negative side is that the people feels more into the political life of the country because it has more influence, and we clearly see its influence in this episode because without the reaction. The Prime Minister therefore tries to follow as much the opinion of the social medias. First, before the finger of the princess is sent, the public supports the Prime Minister, saying that he should not fear the people who threat him. Then, when the finger of the princes is discovered, the public opinion changes and therefore also the Prime Minister's. The idea behind it is that politician are very much under the influence of pleasing the social media trends, they care more about the public criticism than the results, which can sometimes be an issue. But the opposite can also be true. Because of the obsession with instantaneous information, a self-fulfilling prophecy was created. Everyone was so focused on the PM and the pig and not on the life of the princess, that's what forced him into having sex with the pig.
The episode also denounces how fast a picture or a damaging idea/quote can spread on social networks. For example, the plot to change out the PM with the actor was foiled by someone recognising the actor and snapping a picture, immediately uploading it. Is this ability that we have to instantly share things around the world for the sake of nothing but our own personal gratification, is this an ability that we should have?
\\
In our opinion, instantaneousness and social network also multiply the phenomenon of acrasia. Acrasia means the weakness of will. When a car crash happens on the road side, you don't want to watch a bloodshed, but still you take a look. Like the 9/11, everyone watches the PM having sex with the pig on TV, like voyeurism. They shared the video of the ransom, and every bloody details of the news, like the cut finger, and people like it. It remained us as well the treatment of the recent terrorist attacks in France, when journalists endanger operations for the sake of giving information as fast as possible, making their news-show looking like an action TV-show. In this episode, one of the journalist following the case compromises such an operation.


% obsession with polls and to be in line with public opinion
% mediatisation of the politic life and the pressure induced by the social networks
% the public opinion is to blame : they shared the ransom video, watched the PM fuck a pig... => Akrasia 
% like 9/11 live on all TV channels : people like bloody details..


\subsection{15 Million Merits}

Fifteen Million Merits satirises our constant need for entertainment as well as recognition in an alienating society. 
\begin{center}
\includegraphics[width=12cm]{s1e2.jpg}
\end{center}

\textbf{Context}\\ 
In this futuristic society, the proletariat seems to lives inside closed buildings, whose walls are covered with screens. There are several kind of workers : stars, cyclists and cleaners. Most of them are cyclists, who provide energy. The more they pedal, the more they earn merits, that they can use to buy anything. On the walls, advertising is rampant, and the only way to stop it is to pay. While they pedal, they usually watch and pay shows whose stars are either the praised singers, dancers, or porn stars, or the cleaners who are humiliated. 
The worst cyclists become cleaners. However, if they can gather and pay fifteen million merits, they are allowed to compete in a TV-show to be a star.

\textbf{Plot}\\
We follow the life of Bingham Madsen who  has inherited 15 million merits from his dead brother and thus has the luxury of skipping advertisements as often as he likes. Yet one ad keeps on getting his attention : an ad about an 'America's got talent' like show called HotShot. In the same time, Bingham falls in love with a woman, Abi Khan, who dreams about participating to HotShot and becoming a great singer. With his 15 million credits, Bingham pays her a ticket to enter this famous show.\\
After participating to the show, the results for Abi are terrifying : the jury calls her an 'average singer' but proposes her to become a pornographic actress to get the fame she so deeply desires. Abi agrees under the influence of the crowd and a drug which eases her thoughts process.\\
Destroyed by the turn of events, Bingham comes back to his room without any money. There begins a long process in which he swears to gain his 15 million credits and participate to HotShot in order to take revenge upon the members of the jury. After regaining his money and taking a shard of glass with him to harm the members of the jury, Bingham arrives on stage and delivers a breathtaking performance : a tearfull rant about how the society in which he lives has become heartless and unfair. The jury, instead of being targeted by the accusations of Bingham, applauses the beautiful performance Bingham delivered and proposes him to become a star with his own show by appearing every week on television and doing a rant about the topic he just ranted about.\\
Bingham agrees, just like Abi did. Unlike her, he wasn't under any drug's influence. \\
The episode ends on Bingham doing one more time his rant on television while a cyclist is pedalling while watching him. We see Bingham after his performance, inside his new big luxury room, sweeping his anger away and pouring himself a glass of orange juice inside, as if nothing had just happened and as if his words were just nothing but an act.

\begin{center}
\includegraphics[width=12cm]{s1e2b.jpg}
\end{center}

\textbf{Analysis.}

%https://www.reddit.com/r/blackmirror/comments/51zd1z/rewatch_discussion_fifteen_million_merits/?st=j52gxfos&sh=ae2b53ad
This episode is more political than the last one. This episode depicts a world in which the proletariat either stay in a poor living conditions, or work to climb the social ladder. And if they succeed to move up, most of them will do anything not to go down, even if it means giving up their principles, their belief's system. One can say that this episode claims we are ultimately selfish people. But being selfish is putting your needs ahead of others needs. What we have here is the story of a man reevaluating his belief system and deciding what is more important. He can keep his moral high grounds and integrity while pedalling all day long in a box, eating disgusting food with very little control over his sensory inputs, or he can let go of some of that, "sell out," and live in comfort with more freedom. The episode basically asks the question : what share of our belief system are we ready to sacrifice in order to live with more comfort ?\\
We thought that this episode is another critic of the society we live in. At the end of the episode, Bingham performs a rant explaining how unfair this society is and how much he lingers to destroy it. But actually, the jury doesn't take his words for it, but turn his rant into another product that pedalling workers are able to watch while they produce energy. Therefore, society has a tendency to uses what imprints people's minds for its own economic interest even if what is imprinting people's minds is suppose to be working against society. The green agriculture trend taking place in our society, is a good illustration of this. At first it was a response to the intensive and earth-damaging agricultural industry. As big farming companies figured out that green is profitable, they started to make green products of their own. Slowly but surely they manage to take over the green revolution with a monopoly.

\subsection{The Entire History of You}
\textbf{Context}\\
This episode takes place in an alternative world, similar to our world on everything except for the fact that most people, if they choose so, can implant a device in their eyes and their ears which records everything they see and hear, including the best and the worst moments of their lifes. Virtually everyone owns such a device. Its use is simple, you either watch the scenes you have recorded by playing them "in your eyes" (the real world fades away and you are only left with your vision of the past), or you can cast your recordings on any screen. This episode raises the following question : if we were able to remember everything that happened during our entire life, would it be a gift or a curse ? Isn't forgiving a cure to life's deepest wounds ?

\begin{center}
\includegraphics[width=12cm]{s1e3.jpg}
\end{center}

\textbf{Plot}\\
The episode starts with the work interview of Liam, a young lawyer, that he thinks he failed. After leaving the meeting he replays his memory of it and focuses on what he believes is an insincere sentence used by his employer. He repeats the scene, with the devices implanted in his head, over and over again. He arrives at a dinner party hosted by some of his wife's friends and sees his wife Fion talking to a man he doesn't recognise, whom she introduces as Jonas. Some of Fion's friends ask how the appraisal went and suggest replaying it, so they can all give their opinions on it.
\begin{center}
\includegraphics[width=12cm]{bm_roue.jpg}
\end{center}
During the diner, Liam catches what he believes are suspicious looks between his wife and Jonas. Also during the diner, one of the guest explains that her device in her eyes was removed after an accident, and she explains how free she feels now.\\
When Liam and his wife get back to their home, Liam questions her about the relationship between her and Jonas, but she refuses to divulge their entire history to her husband. To reconcile with each other, they have sex but in a peculiar way : they are having "boring" sex but both of them are watching through their eyes a memory of them having "great" sex.
\begin{center}
\includegraphics[width=12cm]{s1e3b.png}
\end{center} Then, after a night getting drunk and watching over and over again the suspicious scenes from the diner, Liam goes to Jonas' place to confront him and force him to delete the scenes he has with his wife. He discovers that Jonas and her had sex, but doesn't watch the scene. After a fight and a car accident on the way back home, Liam finally gets back to Fion and forces her to show him the private sex scenes that Fion and Jonas shared in the past, 18 months before. Liam becomes angry because he wants to know if Fion and Jonas used a condom the night they had sex. After watching the scene, Liam realises that his daughter is actually the daughter of Jonas and Fion. The video he watched proves they did not use a condom.\\
The closing scene takes place in Liam's house, after Fion and her daughter left because Liam became insane. After re-watching some happy memories of him and his daughter, Liam goes to the bathroom and messily cuts into his head's skin to remove the recording device. The screen suddenly goes black.


\textbf{Analysis}\\

\section{Second season}

\subsection{Be Right Back}
\textbf{Plot}\\
A young woman, Martha, moves to a new house with her husband Ash. When he returns the rental van, he pass away in a car crash. Martha, alone, realises she is expecting a child. One of her friend tell her about a new technology that helped her is a similar case. With the content of Ash' social networks, an artificial intelligence can simulate his behaviour. First disgusted, she finally tries it and she is very impressed. At the beginning, she chats only with him through her phone, then with his voice. In fact, the program that simulates him learns everything form his Internet data : videos and content from Facebook, messages and calls from his phone, for example.

Then, her friend tell her that a new experimental feature is available. An humanoid looking, behaving and feeling like him. At first she is very happy with him, and try to do get her former life back. However he is still a robot : he doesn't sleep or eat and he never gets angry.





\subsection{White Bear}
\subsection{White Christmas}






\end{document}

