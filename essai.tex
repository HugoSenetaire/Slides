\documentclass{article}
\usepackage[utf8]{inputenc}
\begin{document}

\maketitle

\section{Effets manuels}
\subsection{Le rôle de l'Aperture}
L'aperture est l'ouverture du diaphragme, permettant de contôler combien de lumière rentre dans l'objectif.

\subsection{Le rôle de l'ISO}
L'ISO est l'acronyme de quelque chose mais personne ne sait. L'ISO va généralement de 250 à 4000.

\subsubsection{Bas ISO}
Les bas ISO sont utilisés lorqu'il y a beaucoup de lumière.

\subsubsection{Haut ISO}
Les hauts iso sont utilisés lorsqu'il y a peu de lumière comme la nuit, les jours nuageux ou encore lorsqu'on a est à l'intérieur d'une demeure.

\subsection{Le rôle du shutter speed}

\subsubsection{Shutter speed : définition}

Le shutter speed est la vitesse de prise d'une photo. Plus cette vitesse est élevée, c'est à dire plus le nombre est élevé, moins il y a de lumière captée.

\subsubsection{Shutter speed : effet de flou}
Penser à mettre un faible shutter speed pour avoir un effet de flou.

\section{Effets digitaux}
\subsection{Adobe Photoshop}
Adobe Photoshop permet de retoucher ses photos

\subsubsection{Adobe Lightroom}
Adobe LightRoom permet de retoucher ses photos.
\subsubsection{Adobe}

\subsection{Les voitures}
Hmmm j'aime les voitures.
\end{document}

