\documentclass{beamer} 
\usetheme{default}
\title{ triva } 

\subject{ } 

 \AtBeginSubsection[] 
 { 
 \begin{frame}<beamer>{Outline} 
    \tableofcontents[currentsection,currentsubsection] 
 \end{frame} 
 } 

\begin{document} 

 
 \begin{frame} 
   \titlepage 
 \end{frame} 
 
 \begin{frame}{Outline} 
 \tableofcontents 
 \end{frame} 

\section{ Filtrage par moyennage non-local : Definition } 
\subsection{Générale} 

  \begin{frame}{Générale}{} 

     \begin{itemize} 
      \item { 
         $$ (G_a * |v(x+.)-v(y + .)|^2)(0) = int_{mathbb{R}^2} (G_a(t) |v(x+t)-v(y+t)|^2) dt $$ 
      } 
   \end{itemize} 
 \end{frame}

\subsection{Norme} 

\section{ Algorithme de moyennage non-local } 
\subsection{Description générale} 

\subsection{Retour sur la norme} 

\section{ Experimentation } 
\subsection{Mise en pratique} 

\subsection{Comparaison} 

\end{document}