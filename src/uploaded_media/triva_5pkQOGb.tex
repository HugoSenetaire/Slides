\documentclass{article}
\usepackage[utf8]{inputenc}
\usepackage{geometry}
\geometry{hmargin=1.5cm,vmargin=1.5cm}

\title{Moyennage non local}
\author{Marius Fouquerand - Louis Montaut }
\date{}
\usepackage{natbib}
\usepackage{graphicx}

\begin{document}

\maketitle

\section{Filtrage par moyennage non-local : Definition} 
\subsection{Générale}
Dans une image, il y a des redondances. Pour réduire le bruit par moyennage non-local, on moyenner l'intensité  d'un point en comparant ce dernier à des points ayant un voisinage similaire. \\
On travaille sur une image $v$ (intensité de l'image) définie sur $\Omega \subset \mathbb{R}^2$. \\
On caractérise les similitudes entre les voisinages de deux points $x$ et $y$ par une fonction $f(x,y)$. On définit alors l'image débruitée $u$ en $x$ ainsi : $$u(x) = \frac{1}{C(x)} \int_{y \in \Omega}f(x,y) v(y) dy  $$ 
\subsection{Norme}
Où C(x) est une constante de normalisation.\\
\\
Un choix classique est de prendre $$f(x,y) = exp(- \frac{(G_a * |v(x+.)-v(y + .)|^2)(0)}{h^2}) $$. 

avec $$ (G_a * |v(x+.)-v(y + .)|^2)(0) = \int_{\mathbb{R}^2} (G_a(t) |v(x+t)-v(y+t)|^2) dt  $$ 

avec $G_a$ un noyau gaussien d'écart-type a. 
\section{Algorithme de moyennage non-local}
\subsection{Description générale}
Passons au cas discret, le cas de la vraie vie, dans lequel nous avons une image $I$ discretisée en pixels.\\
\begin{figure}[h!]
    \centering
    \includegraphics[scale=0.5]{explication.png}
    \caption{Le moyennage local}
\end{figure}
\newpage
On définit le voisinage d'un point $i$, $\mathcal{N}_i \subset I$ de la manière suivante : $\forall i \in I  $
\begin{itemize}
    \item $i \in \mathcal{N}_i$
    \item $i \in \mathcal{N}_j\Rightarrow j \in \mathcal{N}_i$
\end{itemize}
L'algorithme de moyennage non-local calcul pour chaque pixel $i$ d'une image $I$ d'intensité $v$ une valeur de l'instensité de image débruitée $NL[v](i)$, définie de la sorte:
$$NL[v](i) = \sum_{j \in I} w(i,j)v(j)$$

où $$w(i,j) = \frac{1}{Z(i)}exp(\frac{||v(\mathcal{N}_i)-v(\mathcal{N}_{j})||_{2,a}^{2}}{h^{2}})$$
\subsection{Retour sur la norme}
La norme $||v(\mathcal{N}_{i})-v(\mathcal{N}_j)||_{2,a}^{2}$ est la norme euclidienne pondérée par le noyau gaussien d'écart-type a.



\section{Experimentation}
\subsection{Mise en pratique}
\begin{figure}[ht!]
    \centering
    \includegraphics[scale=0.6]{NL}
    \caption{Mise en pratique de l'algorithme sur une photo bruitée}
\end{figure}

\section{Comparaison}
\begin{figure}[ht!]
    \centering
    \includegraphics[scale=0.7]{comparaisons.png}
    \caption{Comparaison de différentes méthodes de débruitage. De haut en bas, de gauche à droite : Image bruitée, Gauss filtering, anisotropic filtering, Total variation, Neighborhood filtering et algorithme de moyennage non-local}
\end{figure}

\section{Une potentielle amélioration avec le filtre bilatéral}
Les problèmes rencontrés sont similaires à ceux de la réduction classique de bruit : les contours, les détails et les structures fines ressemblent à du bruit et peuvent être éliminées par l'algorithme. Dans le cas classique ("moyennage local"), la technique du \textit{filtre bilatéral} apportait une solution à ce problème. On peut donc tout à fait imaginer une solution similaire : dans la définition de $f$ on remplace le noyau gaussien par les deux noyaux définis dans le filtre bilatéral.  
\end{document}

